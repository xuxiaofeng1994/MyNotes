\documentclass{book}

%------------------------------------

\usepackage{CJK}

\usepackage{indentfirst}
\setlength{\parindent}{2em}

\usepackage{setspace}
\linespread{1.5}

\usepackage{listings}
\lstset{
	language={C},
	frame=single,
	frameround=tttt,
	numbers=left,
	numberstyle=\color{gray},
	commentstyle=\color{green},
	keywordstyle=\bf,
	tabsize=2,
	breaklines=true,
	basicstyle=\footnotesize\ttfamily,
	escapeinside=``
}

\usepackage{xcolor}

\usepackage{titlesec}

\usepackage{tocloft}
\setlength{\cftbeforesecskip}{25pt}

\usepackage[
	CJKbookmarks=true,
	pdftex,
	colorlinks,
	linkcolor=black,
	anchorcolor=black,
	citecolor=black]{hyperref}

%--------------------------------------------

\begin{document}
\begin{CJK}{UTF8}{gbsn}

%--------------------------------------------

\title{\Huge\bf开始检索吧!}
\author{\it Kaleo}
\date{}
\maketitle

%--------------------------------------------

\renewcommand{\contentsname}{Keywords Index}
\tableofcontents

%-------------------------------------------

\chapter*{这是我的检索日记}
记录我所遇问题的点点滴滴。。。。
\newpage

%-------------------------------------------

\chapter*{单片机}
\addcontentsline{toc}{chapter}{MCU}
关于单片机,硬件,嵌入式等问题的记录。
\newpage

%------------------------------------------

\section*{52单片机在12Mhz晶振下达到9600Hz波特率串口通信的实现}
\addcontentsline{toc}{section}{52MCU 12Mhz 9600}

51及52单片机默认采用的是T1作为串口方式1和方式3的定时器,但是当单片机的外部振荡晶振为12Mhz的时候,
在高波特率下的误差很大,到9600波特率的时候其误差达到了8.51\%,这么大的误差不可用的,因而我们换一种
定时器T2。T2在52及以上芯片里有,下面是其用法:
\begin{lstlisting}
RCAP2L = 0xD9;
RCAP2H = 0xFF;
//`这两句为波特率为9600Hhz时所对应的RCAP寄存器的初值`

T2CON = 0x34;
//`RCLK = 1,TCLK = 1,TR2 = 1`

SCON = 0x50;
//`串口工作模式1,接收使能`
\end{lstlisting}
初始化上述这段代码,就能在程序中使用9600Hz的串口通信了。
\newpage

%---------------------------------------------
\chapter*{OpenSource}
\addcontentsline{toc}{chapter}{OpenSource}
所有有趣的开源软件的相关问题。

\newpage

%---------------------------------------------

\section*{LaTeX中生成中文目录超链接和目录检索的问题}
\addcontentsline{toc}{section}{LaTeX Chinese Contents Index}

所有有趣的开源软件的相关问题。
在LaTeX中使用中文总有些蛋疼的问题。
要生成中文目录,用hyperref就行了,如下:

\begin{lstlisting}
usepackage[CJKbookmarks=true,
	colorlonks,
	linkcolor=black,
	anchorcolor=black,
	citecolor=black]{hyperref}
\end{lstlisting}

其中第一句是生成目录检索,但是如果没有有这一句,连中文目录的超链接都不能正常的生成。我在这个问题上耗了一下午。。。。。。
这样生成有个蛋疼的问题,中文的目录检索将是乱码,因而我找到了下面这种解决方法:

将CJK包改为CJKutf8包,将引用hyperref包时的CJKbookmarks=true改为unicode=true,顺便再加一句pdftex,如下:

\begin{lstlisting}
usepackage[pdftex,
	unicode=true,
	colorlonks,
	linkcolor=black,
	anchorcolor=black,
	citecolor=black]{hyperref}
\end{lstlisting}

这样的中文索引正常了,可是中文目录超链接乱码了。。。。。

网上还有一种方法,使用什么g...程序将中间文件转码,那个有空再看看吧,先用第一种方案用着吧

\newpage
%---------------------------------------------
\end{CJK}
\end{document}

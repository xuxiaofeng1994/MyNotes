\documentclass{book}

\usepackage{CJK}

\usepackage{indentfirst}
\setlength{\parindent}{2em}

\usepackage{setspace}
\linespread{1.5}

\usepackage{listings}
\lstset{
	language={C},
	frame=single,
	frameround=tttt,
	numbers=left,
	numberstyle=\color{gray},
	commentstyle=\color{green},
	keywordstyle=\bf,
	tabsize=2,
	breaklines=true,
	basicstyle=\footnotesize\ttfamily,
	escapeinside=``
}

\usepackage{xcolor}

\usepackage{titlesec}

\usepackage{tocloft}
\setlength{\cftbeforesecskip}{25pt}

\usepackage[
	CJKbookmarks=true,
	pdftex,
	colorlinks,
	linkcolor=black,
	anchorcolor=black,
	citecolor=black]{hyperref}

\begin{document}
\begin{CJK}{UTF8}{gbsn}
\title{开始检索吧!}
\author{Kaleo}
\date{}
\maketitle
\tableofcontents
\chapter*{这是我的成长日记!}
记录我的点点滴滴。。。。
\newpage
\chapter*{单片机}
\addcontentsline{toc}{chapter}{单片机}

这里记录关于单片机,嵌入式等信息。
\newpage
\section{52单片机在12Mhz晶振下达到9600Hz波特率串口通信的实现}
51及52单片机默认采用的是T1作为串口方式1和方式3的定时器,但是当单片机的外部振荡晶振为12Mhz的时候,
在高波特率下的误差很大,到9600波特率的时候其误差达到了8.51\%,这么大的误差不可用的,因而我们换一种
定时器T2。T2在52及以上芯片里有,下面是其用法:
\begin{lstlisting}
RCAP2L = 0xD9;
RCAP2H = 0xFF;
//`这两句为波特率为9600Hhz时所对应的RCAP寄存器的初值`

T2CON = 0x34;
//`RCLK = 1,TCLK = 1,TR2 = 1`

SCON = 0x50;
//`串口工作模式1,接收使能`
\end{lstlisting}
初始化上述这段代码,就能在程序中使用9600Hz的串口通信了。
\newpage
\end{CJK}
\end{document}

\documentclass{book}

%----------------------------------

\usepackage{CJK}

\usepackage{indentfirst}
\setlength{\parindent}{2em}

\usepackage{setspace}
\linespread{1.5}

\usepackage{listings}
\lstset{
	language={C},
	frame=single,
	frameround=tttt,
	numbers=left,
	numberstyle=\color{gray},
	commentstyle=\color{green},
	keywordstyle=\bf,
	tabsize=2,
	breaklines=true,
	basicstyle=\footnotesize\ttfamily,
	escapeinside=``
}

\usepackage{xcolor}

\usepackage{titlesec}

\usepackage{tocloft}
\setlength{\cftbeforesecskip}{25pt}

\usepackage[
	CJKbookmarks=true,
	pdftex,
	colorlinks,
	linkcolor=black,
	anchorcolor=black,
	citecolor=black]{hyperref}

%--------------------------------------------

\begin{document}
\begin{CJK}{UTF8}{gbsn}

%--------------------------------------------

\title{\Huge\bf开始检索吧!}
\author{\it Kaleo}
\date{}
\maketitle

%--------------------------------------------

\renewcommand{\contentsname}{Keywords Index}
\tableofcontents

%-------------------------------------------

\chapter*{这是我的检索日记}
记录我所遇问题的点点滴滴。。。。
\newpage

%-------------------------------------------

\chapter*{单片机}
\addcontentsline{toc}{chapter}{MCU}
关于单片机,硬件,嵌入式等问题的记录。
\newpage

%------------------------------------------

\section{52MCU 12Mhz Xtal 9600Hz Brid }
%\addcontentsline{toc}{section}{52MCU 12Mhz 9600}

51及52单片机默认采用的是T1作为串口方式1和方式3的定时器,但是当单片机的外部振荡晶振为12Mhz的时候,
在高波特率下的误差很大,到9600波特率的时候其误差达到了8.51\%,这么大的误差不可用的,因而我们换一种
定时器T2。T2在52及以上芯片里有,下面是其用法:
\begin{lstlisting}
RCAP2L = 0xD9;
RCAP2H = 0xFF;
//`这两句为波特率为9600Hz时所对应的RCAP寄存器的初值`

T2CON = 0x34;
//`RCLK = 1,TCLK = 1,TR2 = 1`

SCON = 0x50;
//`串口工作模式1,接收使能`
\end{lstlisting}
初始化上述这段代码,就能在程序中使用9600Hz的串口通信了。
\newpage

%---------------------------------------------

\section{52MCU ultrasonic wave}
%\addcontentsline{toc}{section}{52MCU Ultrasonic Wave}

这个程序本来很简单,但是其中的串口程序让我苦逼了两天,主要是12Mhz的问题(见上一个问题),这个程序的功能是驱动超声波模块測出距离,然后把结果发送到串口上来,代码如下:


\newpage
%---------------------------------------------
\chapter*{OpenSource}
\addcontentsline{toc}{chapter}{OpenSource}
所有有趣的开源软件的相关问题。

\newpage

%---------------------------------------------

\section{LaTeX Chinese Contents Index}
%\addcontentsline{toc}{section}{LaTeX Chinese Contents Index}

在LaTeX中使用中文总有些蛋疼的问题。
要生成中文目录,用hyperref就行了,如下:

\begin{lstlisting}
usepackage[CJKbookmarks=true,
	colorlonks,
	linkcolor=black,
	anchorcolor=black,
	citecolor=black]{hyperref}
\end{lstlisting}

其中第一句是生成目录检索,但是如果没有有这一句,连中文目录的超链接都不能正常的生成。我在这个问题上耗了一下午。。。。。。
这样生成有个蛋疼的问题,中文的目录检索将是乱码,因而我找到了下面这种解决方法:

将CJK包改为CJKutf8包,将引用hyperref包时的CJKbookmarks=true改为unicode=true,顺便再加一句pdftex,如下:

\begin{lstlisting}
usepackage[pdftex,
	unicode=true,
	colorlonks,
	linkcolor=black,
	anchorcolor=black,
	citecolor=black]{hyperref}
\end{lstlisting}

这样的中文索引正常了,可是中文目录超链接乱码了。。。。。

网上还有一种方法,使用什么g...程序将中间文件转码,那个有空再看看吧,先用第一种方案用着吧

\newpage
%---------------------------------------------

\section{matlab install}
%\addcontentsline{toc}{section}{matlab install}

1.解压缩“Mathworks Matlab R2013b Linux.rar”(无需密码),得
到“Mathworks.Matlab.R2013b.Linux.rar”和“Readme.txt”

2.继续解压缩“Mathworks.Matlab.R2013b.Linux.rar”,此时需要密码,密码为“tone”,解压完成之后得到“R2013b\_UNIX.iso”和名为“patch”的破解文件文件夹.

3.继续解压缩“R2013b\_UNIX.iso”到一个文件夹中(我假设这个文件夹名字为R2013b\_UNIX,)然后用破解文件夹中的“patch/Matlab-2013b-Lin64/install\_jar/install.jar”替换“R2013b\_UNIX/java/jar/install.jar”

4.在终端中切换到“R2013b\_UNIX”目录并输入“install”开始安装MATLAB

5.选择不联网安装(withoutinternet),序列号(install key)输入“50099-21292-14201-03250-24790-41816-23907-62461-58657-24048-55364-08874-18566-22288”或者”24301-62136-21555-63628-23951-18882-23524-13990-17158-47212-41804-63025-63883-42443“

6.安装到最后会提示你激活,还是选择”without internet“,然后激活文件选择破解文件夹中的”patch/Matlab-2013b-Lin64/matlab\_std.dat“

7.最后,需要用破解文件夹中的”patch/Matlab-2013b-Lin64/libmwservices.so“替换”/usr/local/MATLAB/R2013b/bin/glnxa64/libmwservices.so

8.在终端输入“/usr/local/MATLAB/R2013b/bin/matlab”开始使用MATLAB

\newpage
%---------------------------------------------

\section{LaTeX install}
%\addcontentsline{toc}{section}{LaTeX install}

1.下载安装包,安装

2.设置环境变量:在/home/.profile中添加这几个环境变量:

\begin{lstlisting}
PATH="opt/texlive/2013/bin/x86_64-linux:$PATH"
MANPATH="opt/texlive/2013/texmf-dist/doc/man:$MANPATH"
INFOPATH="opt/texlive/2013/texmf-dist/doc/info:$INFOPATH"
\end{lstlisting}

%\newpage	
%---------------------------------------------

\section{vim profile}

\begin{lstlisting}
set nu
set hlsearch
set autoindent
set ruler
set showmode
set backspace=2
syntax on
colorscheme elflord
\end{lstlisting}

%\newpage
%---------------------------------------------

\section{LaTeX comment}
\begin{lstlisting}
1. %
2. \if false .....\fi
3. \usepackage{verbatim}
	\begin{comment}
	\end{comment}
\end{lstlisting}

%---------------------------------------------
\end{CJK}
\end{document}
